% Options for packages loaded elsewhere
\PassOptionsToPackage{unicode}{hyperref}
\PassOptionsToPackage{hyphens}{url}
%
\documentclass[
]{article}
\usepackage{amsmath,amssymb}
\usepackage{lmodern}
\usepackage{ifxetex,ifluatex}
\ifnum 0\ifxetex 1\fi\ifluatex 1\fi=0 % if pdftex
  \usepackage[T1]{fontenc}
  \usepackage[utf8]{inputenc}
  \usepackage{textcomp} % provide euro and other symbols
\else % if luatex or xetex
  \usepackage{unicode-math}
  \defaultfontfeatures{Scale=MatchLowercase}
  \defaultfontfeatures[\rmfamily]{Ligatures=TeX,Scale=1}
\fi
% Use upquote if available, for straight quotes in verbatim environments
\IfFileExists{upquote.sty}{\usepackage{upquote}}{}
\IfFileExists{microtype.sty}{% use microtype if available
  \usepackage[]{microtype}
  \UseMicrotypeSet[protrusion]{basicmath} % disable protrusion for tt fonts
}{}
\makeatletter
\@ifundefined{KOMAClassName}{% if non-KOMA class
  \IfFileExists{parskip.sty}{%
    \usepackage{parskip}
  }{% else
    \setlength{\parindent}{0pt}
    \setlength{\parskip}{6pt plus 2pt minus 1pt}}
}{% if KOMA class
  \KOMAoptions{parskip=half}}
\makeatother
\usepackage{xcolor}
\IfFileExists{xurl.sty}{\usepackage{xurl}}{} % add URL line breaks if available
\IfFileExists{bookmark.sty}{\usepackage{bookmark}}{\usepackage{hyperref}}
\hypersetup{
  pdftitle={Contribution},
  hidelinks,
  pdfcreator={LaTeX via pandoc}}
\urlstyle{same} % disable monospaced font for URLs
\usepackage[margin=1in]{geometry}
\usepackage{longtable,booktabs,array}
\usepackage{calc} % for calculating minipage widths
% Correct order of tables after \paragraph or \subparagraph
\usepackage{etoolbox}
\makeatletter
\patchcmd\longtable{\par}{\if@noskipsec\mbox{}\fi\par}{}{}
\makeatother
% Allow footnotes in longtable head/foot
\IfFileExists{footnotehyper.sty}{\usepackage{footnotehyper}}{\usepackage{footnote}}
\makesavenoteenv{longtable}
\usepackage{graphicx}
\makeatletter
\def\maxwidth{\ifdim\Gin@nat@width>\linewidth\linewidth\else\Gin@nat@width\fi}
\def\maxheight{\ifdim\Gin@nat@height>\textheight\textheight\else\Gin@nat@height\fi}
\makeatother
% Scale images if necessary, so that they will not overflow the page
% margins by default, and it is still possible to overwrite the defaults
% using explicit options in \includegraphics[width, height, ...]{}
\setkeys{Gin}{width=\maxwidth,height=\maxheight,keepaspectratio}
% Set default figure placement to htbp
\makeatletter
\def\fps@figure{htbp}
\makeatother
\setlength{\emergencystretch}{3em} % prevent overfull lines
\providecommand{\tightlist}{%
  \setlength{\itemsep}{0pt}\setlength{\parskip}{0pt}}
\setcounter{secnumdepth}{-\maxdimen} % remove section numbering
<!--radix_placeholder_navigation_in_header-->
<meta name="distill:offset" content=""/>

<script type="application/javascript">

  window.headroom_prevent_pin = false;

  window.document.addEventListener("DOMContentLoaded", function (event) {

    // initialize headroom for banner
    var header = $('header').get(0);
    var headerHeight = header.offsetHeight;
    var headroom = new Headroom(header, {
      tolerance: 5,
      onPin : function() {
        if (window.headroom_prevent_pin) {
          window.headroom_prevent_pin = false;
          headroom.unpin();
        }
      }
    });
    headroom.init();
    if(window.location.hash)
      headroom.unpin();
    $(header).addClass('headroom--transition');

    // offset scroll location for banner on hash change
    // (see: https://github.com/WickyNilliams/headroom.js/issues/38)
    window.addEventListener("hashchange", function(event) {
      window.scrollTo(0, window.pageYOffset - (headerHeight + 25));
    });

    // responsive menu
    $('.distill-site-header').each(function(i, val) {
      var topnav = $(this);
      var toggle = topnav.find('.nav-toggle');
      toggle.on('click', function() {
        topnav.toggleClass('responsive');
      });
    });

    // nav dropdowns
    $('.nav-dropbtn').click(function(e) {
      $(this).next('.nav-dropdown-content').toggleClass('nav-dropdown-active');
      $(this).parent().siblings('.nav-dropdown')
         .children('.nav-dropdown-content').removeClass('nav-dropdown-active');
    });
    $("body").click(function(e){
      $('.nav-dropdown-content').removeClass('nav-dropdown-active');
    });
    $(".nav-dropdown").click(function(e){
      e.stopPropagation();
    });
  });
</script>

<style type="text/css">

/* Theme (user-documented overrideables for nav appearance) */

.distill-site-nav {
  color: rgba(255, 255, 255, 0.8);
  background-color: #0F2E3D;
  font-size: 15px;
  font-weight: 300;
}

.distill-site-nav a {
  color: inherit;
  text-decoration: none;
}

.distill-site-nav a:hover {
  color: white;
}

@media print {
  .distill-site-nav {
    display: none;
  }
}

.distill-site-header {

}

.distill-site-footer {

}


/* Site Header */

.distill-site-header {
  width: 100%;
  box-sizing: border-box;
  z-index: 3;
}

.distill-site-header .nav-left {
  display: inline-block;
  margin-left: 8px;
}

@media screen and (max-width: 768px) {
  .distill-site-header .nav-left {
    margin-left: 0;
  }
}


.distill-site-header .nav-right {
  float: right;
  margin-right: 8px;
}

.distill-site-header a,
.distill-site-header .title {
  display: inline-block;
  text-align: center;
  padding: 14px 10px 14px 10px;
}

.distill-site-header .title {
  font-size: 18px;
  min-width: 150px;
}

.distill-site-header .logo {
  padding: 0;
}

.distill-site-header .logo img {
  display: none;
  max-height: 20px;
  width: auto;
  margin-bottom: -4px;
}

.distill-site-header .nav-image img {
  max-height: 18px;
  width: auto;
  display: inline-block;
  margin-bottom: -3px;
}



@media screen and (min-width: 1000px) {
  .distill-site-header .logo img {
    display: inline-block;
  }
  .distill-site-header .nav-left {
    margin-left: 20px;
  }
  .distill-site-header .nav-right {
    margin-right: 20px;
  }
  .distill-site-header .title {
    padding-left: 12px;
  }
}


.distill-site-header .nav-toggle {
  display: none;
}

.nav-dropdown {
  display: inline-block;
  position: relative;
}

.nav-dropdown .nav-dropbtn {
  border: none;
  outline: none;
  color: rgba(255, 255, 255, 0.8);
  padding: 16px 10px;
  background-color: transparent;
  font-family: inherit;
  font-size: inherit;
  font-weight: inherit;
  margin: 0;
  margin-top: 1px;
  z-index: 2;
}

.nav-dropdown-content {
  display: none;
  position: absolute;
  background-color: white;
  min-width: 200px;
  border: 1px solid rgba(0,0,0,0.15);
  border-radius: 4px;
  box-shadow: 0px 8px 16px 0px rgba(0,0,0,0.1);
  z-index: 1;
  margin-top: 2px;
  white-space: nowrap;
  padding-top: 4px;
  padding-bottom: 4px;
}

.nav-dropdown-content hr {
  margin-top: 4px;
  margin-bottom: 4px;
  border: none;
  border-bottom: 1px solid rgba(0, 0, 0, 0.1);
}

.nav-dropdown-active {
  display: block;
}

.nav-dropdown-content a, .nav-dropdown-content .nav-dropdown-header {
  color: black;
  padding: 6px 24px;
  text-decoration: none;
  display: block;
  text-align: left;
}

.nav-dropdown-content .nav-dropdown-header {
  display: block;
  padding: 5px 24px;
  padding-bottom: 0;
  text-transform: uppercase;
  font-size: 14px;
  color: #999999;
  white-space: nowrap;
}

.nav-dropdown:hover .nav-dropbtn {
  color: white;
}

.nav-dropdown-content a:hover {
  background-color: #ddd;
  color: black;
}

.nav-right .nav-dropdown-content {
  margin-left: -45%;
  right: 0;
}

@media screen and (max-width: 768px) {
  .distill-site-header a, .distill-site-header .nav-dropdown  {display: none;}
  .distill-site-header a.nav-toggle {
    float: right;
    display: block;
  }
  .distill-site-header .title {
    margin-left: 0;
  }
  .distill-site-header .nav-right {
    margin-right: 0;
  }
  .distill-site-header {
    overflow: hidden;
  }
  .nav-right .nav-dropdown-content {
    margin-left: 0;
  }
}


@media screen and (max-width: 768px) {
  .distill-site-header.responsive {position: relative; min-height: 500px; }
  .distill-site-header.responsive a.nav-toggle {
    position: absolute;
    right: 0;
    top: 0;
  }
  .distill-site-header.responsive a,
  .distill-site-header.responsive .nav-dropdown {
    display: block;
    text-align: left;
  }
  .distill-site-header.responsive .nav-left,
  .distill-site-header.responsive .nav-right {
    width: 100%;
  }
  .distill-site-header.responsive .nav-dropdown {float: none;}
  .distill-site-header.responsive .nav-dropdown-content {position: relative;}
  .distill-site-header.responsive .nav-dropdown .nav-dropbtn {
    display: block;
    width: 100%;
    text-align: left;
  }
}

/* Site Footer */

.distill-site-footer {
  width: 100%;
  overflow: hidden;
  box-sizing: border-box;
  z-index: 3;
  margin-top: 30px;
  padding-top: 30px;
  padding-bottom: 30px;
  text-align: center;
}

/* Headroom */

d-title {
  padding-top: 6rem;
}

@media print {
  d-title {
    padding-top: 4rem;
  }
}

.headroom {
  z-index: 1000;
  position: fixed;
  top: 0;
  left: 0;
  right: 0;
}

.headroom--transition {
  transition: all .4s ease-in-out;
}

.headroom--unpinned {
  top: -100px;
}

.headroom--pinned {
  top: 0;
}

/* adjust viewport for navbar height */
/* helps vertically center bootstrap (non-distill) content */
.min-vh-100 {
  min-height: calc(100vh - 100px) !important;
}

</style>

<script src="site_libs/jquery-3.6.0/jquery-3.6.0.min.js"></script>
<link href="site_libs/font-awesome-5.1.0/css/all.css" rel="stylesheet"/>
<link href="site_libs/font-awesome-5.1.0/css/v4-shims.css" rel="stylesheet"/>
<script src="site_libs/headroom-0.9.4/headroom.min.js"></script>
<script src="site_libs/autocomplete-0.37.1/autocomplete.min.js"></script>
<script src="site_libs/fuse-6.4.1/fuse.min.js"></script>

<script type="application/javascript">

function getMeta(metaName) {
  var metas = document.getElementsByTagName('meta');
  for (let i = 0; i < metas.length; i++) {
    if (metas[i].getAttribute('name') === metaName) {
      return metas[i].getAttribute('content');
    }
  }
  return '';
}

function offsetURL(url) {
  var offset = getMeta('distill:offset');
  return offset ? offset + '/' + url : url;
}

function createFuseIndex() {

  // create fuse index
  var options = {
    keys: [
      { name: 'title', weight: 20 },
      { name: 'categories', weight: 15 },
      { name: 'description', weight: 10 },
      { name: 'contents', weight: 5 },
    ],
    ignoreLocation: true,
    threshold: 0
  };
  var fuse = new window.Fuse([], options);

  // fetch the main search.json
  return fetch(offsetURL('search.json'))
    .then(function(response) {
      if (response.status == 200) {
        return response.json().then(function(json) {
          // index main articles
          json.articles.forEach(function(article) {
            fuse.add(article);
          });
          // download collections and index their articles
          return Promise.all(json.collections.map(function(collection) {
            return fetch(offsetURL(collection)).then(function(response) {
              if (response.status === 200) {
                return response.json().then(function(articles) {
                  articles.forEach(function(article) {
                    fuse.add(article);
                  });
                })
              } else {
                return Promise.reject(
                  new Error('Unexpected status from search index request: ' +
                            response.status)
                );
              }
            });
          })).then(function() {
            return fuse;
          });
        });

      } else {
        return Promise.reject(
          new Error('Unexpected status from search index request: ' +
                      response.status)
        );
      }
    });
}

window.document.addEventListener("DOMContentLoaded", function (event) {

  // get search element (bail if we don't have one)
  var searchEl = window.document.getElementById('distill-search');
  if (!searchEl)
    return;

  createFuseIndex()
    .then(function(fuse) {

      // make search box visible
      searchEl.classList.remove('hidden');

      // initialize autocomplete
      var options = {
        autoselect: true,
        hint: false,
        minLength: 2,
      };
      window.autocomplete(searchEl, options, [{
        source: function(query, callback) {
          const searchOptions = {
            isCaseSensitive: false,
            shouldSort: true,
            minMatchCharLength: 2,
            limit: 10,
          };
          var results = fuse.search(query, searchOptions);
          callback(results
            .map(function(result) { return result.item; })
          );
        },
        templates: {
          suggestion: function(suggestion) {
            var img = suggestion.preview && Object.keys(suggestion.preview).length > 0
              ? `<img src="${offsetURL(suggestion.preview)}"</img>`
              : '';
            var html = `
              <div class="search-item">
                <h3>${suggestion.title}</h3>
                <div class="search-item-description">
                  ${suggestion.description || ''}
                </div>
                <div class="search-item-preview">
                  ${img}
                </div>
              </div>
            `;
            return html;
          }
        }
      }]).on('autocomplete:selected', function(event, suggestion) {
        window.location.href = offsetURL(suggestion.path);
      });
      // remove inline display style on autocompleter (we want to
      // manage responsive display via css)
      $('.algolia-autocomplete').css("display", "");
    })
    .catch(function(error) {
      console.log(error);
    });

});

</script>

<style type="text/css">

.nav-search {
  font-size: x-small;
}

/* Algolioa Autocomplete */

.algolia-autocomplete {
  display: inline-block;
  margin-left: 10px;
  vertical-align: sub;
  background-color: white;
  color: black;
  padding: 6px;
  padding-top: 8px;
  padding-bottom: 0;
  border-radius: 6px;
  border: 1px #0F2E3D solid;
  width: 180px;
}


@media screen and (max-width: 768px) {
  .distill-site-nav .algolia-autocomplete {
    display: none;
    visibility: hidden;
  }
  .distill-site-nav.responsive .algolia-autocomplete {
    display: inline-block;
    visibility: visible;
  }
  .distill-site-nav.responsive .algolia-autocomplete .aa-dropdown-menu {
    margin-left: 0;
    width: 400px;
    max-height: 400px;
  }
}

.algolia-autocomplete .aa-input, .algolia-autocomplete .aa-hint {
  width: 90%;
  outline: none;
  border: none;
}

.algolia-autocomplete .aa-hint {
  color: #999;
}
.algolia-autocomplete .aa-dropdown-menu {
  width: 550px;
  max-height: 70vh;
  overflow-x: visible;
  overflow-y: scroll;
  padding: 5px;
  margin-top: 3px;
  margin-left: -150px;
  background-color: #fff;
  border-radius: 5px;
  border: 1px solid #999;
  border-top: none;
}

.algolia-autocomplete .aa-dropdown-menu .aa-suggestion {
  cursor: pointer;
  padding: 5px 4px;
  border-bottom: 1px solid #eee;
}

.algolia-autocomplete .aa-dropdown-menu .aa-suggestion:last-of-type {
  border-bottom: none;
  margin-bottom: 2px;
}

.algolia-autocomplete .aa-dropdown-menu .aa-suggestion .search-item {
  overflow: hidden;
  font-size: 0.8em;
  line-height: 1.4em;
}

.algolia-autocomplete .aa-dropdown-menu .aa-suggestion .search-item h3 {
  font-size: 1rem;
  margin-block-start: 0;
  margin-block-end: 5px;
}

.algolia-autocomplete .aa-dropdown-menu .aa-suggestion .search-item-description {
  display: inline-block;
  overflow: hidden;
  height: 2.8em;
  width: 80%;
  margin-right: 4%;
}

.algolia-autocomplete .aa-dropdown-menu .aa-suggestion .search-item-preview {
  display: inline-block;
  width: 15%;
}

.algolia-autocomplete .aa-dropdown-menu .aa-suggestion .search-item-preview img {
  height: 3em;
  width: auto;
  display: none;
}

.algolia-autocomplete .aa-dropdown-menu .aa-suggestion .search-item-preview img[src] {
  display: initial;
}

.algolia-autocomplete .aa-dropdown-menu .aa-suggestion.aa-cursor {
  background-color: #eee;
}
.algolia-autocomplete .aa-dropdown-menu .aa-suggestion em {
  font-weight: bold;
  font-style: normal;
}

</style>


<!--/radix_placeholder_navigation_in_header-->
<!--radix_placeholder_site_in_header-->
<!--/radix_placeholder_site_in_header-->

<style type="text/css">
body {
  padding-top: 60px;
}
</style>
<style type="text/css">
/* base variables */

/* Edit the CSS properties in this file to create a custom
   Distill theme. Only edit values in the right column
   for each row; values shown are the CSS defaults.
   To return any property to the default,
   you may set its value to: unset
   All rows must end with a semi-colon.                      */

/* Optional: embed custom fonts here with `@import`          */
/* This must remain at the top of this file.                 */



html {
  /*-- Main font sizes --*/
  --title-size:      50px;
  --body-size:       1.06rem;
  --code-size:       14px;
  --aside-size:      12px;
  --fig-cap-size:    13px;
  /*-- Main font colors --*/
  --title-color:     #000000;
  --header-color:    rgba(0, 0, 0, 0.8);
  --body-color:      rgba(0, 0, 0, 0.8);
  --aside-color:     rgba(0, 0, 0, 0.6);
  --fig-cap-color:   rgba(0, 0, 0, 0.6);
  /*-- Specify custom fonts ~~~ must be imported above   --*/
  --heading-font:    sans-serif;
  --mono-font:       monospace;
  --body-font:       sans-serif;
  --navbar-font:     sans-serif;  /* websites + blogs only */
}

/*-- ARTICLE METADATA --*/
d-byline {
  --heading-size:    0.6rem;
  --heading-color:   rgba(0, 0, 0, 0.5);
  --body-size:       0.8rem;
  --body-color:      rgba(0, 0, 0, 0.8);
}

/*-- ARTICLE TABLE OF CONTENTS --*/
.d-contents {
  --heading-size:    18px;
  --contents-size:   13px;
}

/*-- ARTICLE APPENDIX --*/
d-appendix {
  --heading-size:    15px;
  --heading-color:   rgba(0, 0, 0, 0.65);
  --text-size:       0.8em;
  --text-color:      rgba(0, 0, 0, 0.5);
}

/*-- WEBSITE HEADER + FOOTER --*/
/* These properties only apply to Distill sites and blogs  */

.distill-site-header {
  --title-size:       18px;
  --text-color:       rgba(255, 255, 255, 0.8);
  --text-size:        15px;
  --hover-color:      white;
  --bkgd-color:       #0F2E3D;
}

.distill-site-footer {
  --text-color:       rgba(255, 255, 255, 0.8);
  --text-size:        15px;
  --hover-color:      white;
  --bkgd-color:       #0F2E3D;
}

/*-- Additional custom styles --*/
/* Add any additional CSS rules below                      */
</style>
<style type="text/css">
/* base variables */

/* Edit the CSS properties in this file to create a custom
   Distill theme. Only edit values in the right column
   for each row; values shown are the CSS defaults.
   To return any property to the default,
   you may set its value to: unset
   All rows must end with a semi-colon.                      */

/* Optional: embed custom fonts here with `@import`          */
/* This must remain at the top of this file.                 */
@import url('https://fonts.googleapis.com/css2?family=Open+Sans');
@import url('https://fonts.googleapis.com/css2?family=Lora');
@import url('https://fonts.googleapis.com/css2?family=Inconsolata');
@import url('https://fonts.googleapis.com/css2?family=Overlock+SC');


html {
  /*-- Main font sizes --*/
  --title-size:      50px;
  --body-size:       1.06rem;
  --code-size:       14px;
  --aside-size:      12px;
  --fig-cap-size:    13px;
  /*-- Main font colors --*/
  --title-color:     #57590C;
  --header-color:    #222601;
  --body-color:      rgba(0, 0, 0, 0.8);
  --aside-color:     rgba(0, 0, 0, 0.6);
  --fig-cap-color:   rgba(0, 0, 0, 0.6);
  /*-- Specify custom fonts ~~~ must be imported above   --*/
  --heading-font:    "Open Sans", sans-serif;
  --mono-font:       "Inconsolata", monospace;
  --body-font:       "Lora", serif;
  --navbar-font:     "Overlock SC", sans-serif;  /* websites + blogs only */
}

/*-- ARTICLE METADATA --*/
d-byline {
  --heading-size:    0.6rem;
  --heading-color:   rgba(0, 0, 0, 0.5);
  --body-size:       0.8rem;
  --body-color:      rgba(0, 0, 0, 0.8);
}

/*-- ARTICLE TABLE OF CONTENTS --*/
.d-contents {
  --heading-size:    18px;
  --contents-size:   13px;
}

/*-- ARTICLE APPENDIX --*/
d-appendix {
  --heading-size:    15px;
  --heading-color:   rgba(0, 0, 0, 0.65);
  --text-size:       0.8em;
  --text-color:      rgba(0, 0, 0, 0.5);
}

/*-- WEBSITE HEADER + FOOTER --*/
/* These properties only apply to Distill sites and blogs  */

.distill-site-header {
   padding: 10px 0px;
  --title-size:       20px;
  --text-color:       #88A61C;
  --text-size:        20px;
  --hover-color:      white;
  --bkgd-color:       #222601;
}

.distill-site-footer {
  --text-color:       rgba(255, 255, 255, 0.8);
  --text-size:        15px;
  --hover-color:      white;
  --bkgd-color:       #0F2E3D;
}

/*-- Additional custom styles --*/
/* Add any additional CSS rules below                      */
.iconlink0 {
    background-color: #222601;
    color: #f1f2eb;
    padding: 3px 5px 3px 5px;
    margin: 0 2px 0 2px;
    border-radius: 5px; /* Rounded edges */
}
.iconlink1 {
    background-color: #222601;
    color: #88A61C;
    padding: 3px 5px 3px 5px;
    margin: 0 2px 0 2px;
    border-radius: 5px; /* Rounded edges */
}
.iconlink2 {
    background-color: #222601;
    color: #D5D977;
    padding: 3px 5px 3px 5px;
    margin: 0 2px 0 2px;
    border-radius: 5px; /* Rounded edges */
}

.iconlink3 {
    background-color: #222601;
    color: #ad6705;
    padding: 3px 5px 3px 5px;
    margin: 0 2px 0 2px;
    border-radius: 5px; /* Rounded edges */
}
.iconlink0:hover {
    background-color: #D5D977;
    color: #222601;
}
.iconlink1:hover {
    background-color: #D5D977;
    color: #222601;
}
.iconlink2:hover {
    background-color: #D5D977;
    color: #222601;
}
.iconlink3:hover {
    background-color: #D5D977;
    color: #222601;
}
</style>
<style type="text/css">
/* base style */

/* FONT FAMILIES */

:root {
  --heading-default: -apple-system, BlinkMacSystemFont, "Segoe UI", Roboto, Oxygen, Ubuntu, Cantarell, "Fira Sans", "Droid Sans", "Helvetica Neue", Arial, sans-serif;
  --mono-default: Consolas, Monaco, 'Andale Mono', 'Ubuntu Mono', monospace;
  --body-default: -apple-system, BlinkMacSystemFont, "Segoe UI", Roboto, Oxygen, Ubuntu, Cantarell, "Fira Sans", "Droid Sans", "Helvetica Neue", Arial, sans-serif;
}

body,
.posts-list .post-preview p,
.posts-list .description p {
  font-family: var(--body-font), var(--body-default);
}

h1, h2, h3, h4, h5, h6,
.posts-list .post-preview h2,
.posts-list .description h2 {
  font-family: var(--heading-font), var(--heading-default);
}

d-article div.sourceCode code,
d-article pre code {
  font-family: var(--mono-font), var(--mono-default);
}


/*-- TITLE --*/
d-title h1,
.posts-list > h1 {
  color: var(--title-color, black);
}

d-title h1 {
  font-size: var(--title-size, 50px);
}

/*-- HEADERS --*/
d-article h1,
d-article h2,
d-article h3,
d-article h4,
d-article h5,
d-article h6 {
  color: var(--header-color, rgba(0, 0, 0, 0.8));
}

/*-- BODY --*/
d-article > p,  /* only text inside of <p> tags */
d-article > ul, /* lists */
d-article > ol {
  color: var(--body-color, rgba(0, 0, 0, 0.8));
  font-size: var(--body-size, 1.06rem);
}


/*-- CODE --*/
d-article div.sourceCode code,
d-article pre code {
  font-size: var(--code-size, 14px);
}

/*-- ASIDE --*/
d-article aside {
  font-size: var(--aside-size, 12px);
  color: var(--aside-color, rgba(0, 0, 0, 0.6));
}

/*-- FIGURE CAPTIONS --*/
figure .caption,
figure figcaption,
.figure .caption {
  font-size: var(--fig-cap-size, 13px);
  color: var(--fig-cap-color, rgba(0, 0, 0, 0.6));
}

/*-- METADATA --*/
d-byline h3 {
  font-size: var(--heading-size, 0.6rem);
  color: var(--heading-color, rgba(0, 0, 0, 0.5));
}

d-byline {
  font-size: var(--body-size, 0.8rem);
  color: var(--body-color, rgba(0, 0, 0, 0.8));
}

d-byline a,
d-article d-byline a {
  color: var(--body-color, rgba(0, 0, 0, 0.8));
}

/*-- TABLE OF CONTENTS --*/
.d-contents nav h3 {
  font-size: var(--heading-size, 18px);
}

.d-contents nav a {
  font-size: var(--contents-size, 13px);
}

/*-- APPENDIX --*/
d-appendix h3 {
  font-size: var(--heading-size, 15px);
  color: var(--heading-color, rgba(0, 0, 0, 0.65));
}

d-appendix {
  font-size: var(--text-size, 0.8em);
  color: var(--text-color, rgba(0, 0, 0, 0.5));
}

d-appendix d-footnote-list a.footnote-backlink {
  color: var(--text-color, rgba(0, 0, 0, 0.5));
}

/*-- WEBSITE HEADER + FOOTER --*/
.distill-site-header .title {
  font-size: var(--title-size, 18px);
  font-family: var(--navbar-font), var(--heading-default);
}

.distill-site-header a,
.nav-dropdown .nav-dropbtn {
  font-family: var(--navbar-font), var(--heading-default);
}

.nav-dropdown .nav-dropbtn {
  color: var(--text-color, rgba(255, 255, 255, 0.8));
  font-size: var(--text-size, 15px);
}

.distill-site-header a:hover,
.nav-dropdown:hover .nav-dropbtn {
  color: var(--hover-color, white);
}

.distill-site-header {
  font-size: var(--text-size, 15px);
  color: var(--text-color, rgba(255, 255, 255, 0.8));
  background-color: var(--bkgd-color, #0F2E3D);
}

.distill-site-footer {
  font-size: var(--text-size, 15px);
  color: var(--text-color, rgba(255, 255, 255, 0.8));
  background-color: var(--bkgd-color, #0F2E3D);
}

.distill-site-footer a:hover {
  color: var(--hover-color, white);
}</style>
\ifluatex
  \usepackage{selnolig}  % disable illegal ligatures
\fi

\title{Contribution}
\author{}
\date{\vspace{-2.5em}}

\begin{document}
\maketitle

<!--radix_placeholder_navigation_before_body-->
<header class="header header--fixed" role="banner">
<nav class="distill-site-nav distill-site-header">
<div class="nav-left">
<a href="index.html" class="title">Learn EcoStat</a>
<div class="nav-dropdown">
<button class="nav-dropbtn">
BASICS
 
<span class="down-arrow">&#x25BE;</span>
</button>
<div class="nav-dropdown-content">
<a href="basics_r.html">R</a>
<a href="basics_ethics.html">Ethics</a>
<a href="basics_data.html">Data</a>
<a href="basics_stats.html">Stats</a>
<a href="basics_maths.html">Maths</a>
</div>
</div>
<div class="nav-dropdown">
<button class="nav-dropbtn">
MODELING
 
<span class="down-arrow">&#x25BE;</span>
</button>
<div class="nav-dropdown-content">
<a href="modeling_regression.html">Regression</a>
<a href="modeling_bayesian.html">Bayesian Approach</a>
<a href="modeling_multivariate.html">Multivariate Analyses</a>
<a href="modeling_machinelearning.html">Machine Learning</a>
</div>
</div>
<div class="nav-dropdown">
<button class="nav-dropbtn">
byFIELD
 
<span class="down-arrow">&#x25BE;</span>
</button>
<div class="nav-dropdown-content">
<a href="byfield_demography.html">Demography</a>
<a href="byfield_population.html">Population Dynamics</a>
<a href="byfield_spatial.html">Spatial Analyses</a>
<a href="byfield_community.html">Community Analyses</a>
</div>
</div>
<input id="distill-search" class="nav-search hidden" type="text" placeholder="Search..."/>
</div>
<div class="nav-right">
<a href="others.html">Others</a>
<a href="forms.html">Contribute</a>
<a href="about.html">About</a>
<a href="https://github.com/fplard/teaching_resources.git">
<i class="fab fa-github" aria-hidden="true"></i>
</a>
<a href="javascript:void(0);" class="nav-toggle">&#9776;</a>
</div>
</nav>
</header>
<!--/radix_placeholder_navigation_before_body-->

<!--radix_placeholder_site_before_body-->
<!--/radix_placeholder_site_before_body-->

If you are in charge of a page, you should appear in the first section
of this document and should have been granted with access to the source
code. If not, please send an email to
\href{mailto:floriane.c.plard@gmail.com}{\nolinkurl{floriane.c.plard@gmail.com}}.

\hypertarget{organisation-of-the-website}{%
\section{Organisation of the
website}\label{organisation-of-the-website}}

\hypertarget{site-plan}{%
\subsection{site plan}\label{site-plan}}

\hypertarget{basics}{%
\subsubsection{Basics}\label{basics}}

\hypertarget{r-vincent-miele}{%
\subparagraph{\texorpdfstring{\href{basics_r.html}{R}: Vincent
Miele?}{R: Vincent Miele?}}\label{r-vincent-miele}}

\emph{contribute: I think there is already enough online on basic stuff
but we need more advanced resources about coding}

\hypertarget{ethics-sandra-hamel}{%
\subparagraph{\texorpdfstring{\href{basics_ethics.html}{Ethics}:
\textbf{Sandra
Hamel}}{Ethics: Sandra Hamel}}\label{ethics-sandra-hamel}}

\emph{contribute: V Amrhein, A Gelman, S Nakagawa
(\href{https://www.sortee.org/people/}{Founding Members of Sortee}), }

\hypertarget{data}{%
\subparagraph{\texorpdfstring{\href{basics_data.html}{Data}:}{Data:}}\label{data}}

\emph{contribute: already a lot of things to add from the web}

\hypertarget{stats-gilles-yoccoz}{%
\subparagraph{\texorpdfstring{\href{basics_stats.html}{Stats}:
\textbf{Gilles
Yoccoz}}{Stats: Gilles Yoccoz}}\label{stats-gilles-yoccoz}}

\hypertarget{maths-fred}{%
\subparagraph{\texorpdfstring{\href{basics_maths.html}{Maths}:
Fred?}{Maths: Fred?}}\label{maths-fred}}

\emph{contribute: J Matthiopoulos, B McGill, S Otto, F Débarre, Lyon
teachers?,
\href{https://www.zoo.ox.ac.uk/people/professor-michael-bonsall?filter_types-1860401\%5B\%5D=\&filter_series-1860401\%5B\%5D=\#tab-1860396}{M
Bonsall}}

There are things online but I did not find really ecologist friendly
courses so far.

\hypertarget{modeling}{%
\subsubsection{Modeling}\label{modeling}}

\hypertarget{regression-gilles-yoccoz}{%
\subparagraph{\texorpdfstring{\href{modeling_regression.html}{Regression}:
\textbf{Gilles
Yoccoz}}{Regression: Gilles Yoccoz}}\label{regression-gilles-yoccoz}}

\emph{contribute: RA Poldrack, Mickael Clark, William A Link,
\href{https://qcnr.usu.edu/directory/adler_peter}{Peter Adler}, B
Bolker, M Brooks, JD Hadfield, L Crespin, D Westneat, G Simpson, B
Shipley, J Lefcheck, PB Conn, M Hooten, N T Hobbs, C Dormann, F Hartig}

\hypertarget{bayesian-approach-olivier-gimenez}{%
\subparagraph{\texorpdfstring{\href{modeling_bayesian.html}{Bayesian
Approach}: \textbf{Olivier
Gimenez}}{Bayesian Approach: Olivier Gimenez}}\label{bayesian-approach-olivier-gimenez}}

\emph{contribute: R McElreath, F. Korner, P deValpine, T Hobbs}

Already some very good courses in this part and a lot of resources that
can be indexed in the more specific folders from these courses \& the
web.

\hypertarget{multivariate-analyses-stuxe9phane-dray}{%
\subparagraph{\texorpdfstring{\href{modeling_multivariate.html}{Multivariate
Analyses}: \textbf{Stéphane
Dray}}{Multivariate Analyses: Stéphane Dray}}\label{multivariate-analyses-stuxe9phane-dray}}

\emph{contribute: S Jadhav, M Palmer, F Husson, AB Dufour, J Thioulouse,
P Legendre, P Peres-Neto, M Anderson, S Holmes, C ter Braak }

\hypertarget{machine-learning}{%
\subparagraph{\texorpdfstring{\href{modeling_machinelearning.html}{Machine
Learning}:}{Machine Learning:}}\label{machine-learning}}

\emph{contribute: JL Parouty, L Jacob? }

wrote a book: \emph{Grant Humphries, Dawn R. Magness, Falk Huettmann,
Pablo Casas}

wrote scientific articles as first or last author about application in
Ecology: \emph{\href{https://timcdlucas.github.io/}{Tim C D Lucas},
André C Ferreira, Sylvain Christin, Nicolas Lecomte, Marek Borowiec,
Alexander E white, Michael A Tabak, Ryan S. Miller}

wrote blog about deep learning for Ecology: \emph{Patrick Gray,
\href{https://jonlefcheck.net/}{J Lefneck}}

\hypertarget{byfield}{%
\subsubsection{byField}\label{byfield}}

\hypertarget{demography-sarah-cubaynes}{%
\subparagraph{\texorpdfstring{\href{byfield_demography.html}{Demography}:
\textbf{Sarah
Cubaynes}}{Demography: Sarah Cubaynes}}\label{demography-sarah-cubaynes}}

\emph{contribute: M Gamelon, C Bonenfant, S Bouwhuis, O Jones,
\href{https://www.kent.ac.uk/mathematics-statistics-actuarial-science/people/1039/matechou-eleni}{E.
Matechou}, F. colchero, R Chandler, M Schaub, G Péron, B Morgan, RS
McCrea, D Fletcher, J Rotella}

\hypertarget{population-dynamics-olivier-gimenez}{%
\subparagraph{\texorpdfstring{\href{byfield_population.html}{Population
Dynamics}: \textbf{Olivier
Gimenez}}{Population Dynamics: Olivier Gimenez}}\label{population-dynamics-olivier-gimenez}}

\emph{contribute: C Nater, E Simmonds,
\href{http://canuck.dnr.cornell.edu/teaching/}{Evan Cooch}, H Caswell, Y
Vindenes, S Engen, A Lee, V Grotan, B O'Hara, D Koons, BE Saether, T
Coulson, S Ellner, D Childs, M Schaub, T Besbeas, E Zipkin, V Radchuk,
A. Kuparinen, M Paniw, C Coste, P Klepac, CJ Metcalf, A de Roos, T van
Dooren}

\hypertarget{spatial-analyses}{%
\subparagraph{\texorpdfstring{\href{byfield_spatial.html}{Spatial
Analyses}:}{Spatial Analyses:}}\label{spatial-analyses}}

\emph{contribute: G Péron, CJ Brown, A Royle, M Kéry, E Zipkin, B
Gardner, J Lahoz-Monfort,
\href{https://fieberg-lab.cfans.umn.edu/people/john-fieberg}{J Fieberg},
R Kays, R.
\href{https://www.uni-bielefeld.de/fakultaeten/wirtschaftswissenschaften/lehrbereiche/stats/team/prof.-dr.-roland-langrock/}{Langrock},
B McClintock, N Ranc}

\hypertarget{community-analyses-d-zeleny}{%
\subparagraph{\texorpdfstring{\href{byfield_community.html}{Community
Analyses}: D Zeleny
?}{Community Analyses: D Zeleny ?}}\label{community-analyses-d-zeleny}}

\emph{contribute: D Borcard, F Gillet, D Zelený, G Blanchet, J Oksanen,
GL Simpson, D. Roberts, Otso Ovoskainen, Francis Hui, David Warton}

\hypertarget{cross-species-analysis}{%
\subparagraph{Cross-species Analysis:}\label{cross-species-analysis}}

\emph{contribute: JF Lemaitre, S Nakagawa, V Ronget} Meta-Analysis,
Allometric models, Phylogeny

\hypertarget{quantitative-genetics-p-de-villemereuil}{%
\subparagraph{Quantitative Genetics: P De Villemereuil
?}\label{quantitative-genetics-p-de-villemereuil}}

\emph{contribute: LM Chevin, A Kuparinen, B Walsh, M Morrissey, T
Bonnet}

Already \href{https://www.youtube.com/watch?v=U-MTfNw7IvM}{courses} from
B Walsh that can be added

\hypertarget{network-analysis}{%
\subparagraph{Network Analysis:}\label{network-analysis}}

\emph{contribute: }

\hypertarget{organisation-of-the-website-1}{%
\subsection{Organisation of the
website}\label{organisation-of-the-website-1}}

Resources are classified using three main parts in the website: Basics,
Modeling, and byField. Each part contains several pages, one per topic.

The website was created using distill website and Rmarkdown. Each page
is a Rmarkdown page named ``partname\_topicname.Rmd''. For instance, the
page R in the part Basics can be edited from \emph{basics\_r.Rmd}. You
can open this file and use it as an example.

There is one page named \href{others.html}{others}. This page can be
used to add interesting links that we do not know where to include.

\hypertarget{organisation-of-each-page}{%
\subsection{Organisation of each page}\label{organisation-of-each-page}}

Each page is written in Rmarkdown and rendered into a webpage using
\emph{render\_site()} from the package rmarkdown.

Each page must be divided into informative sections. Sections are
separated by the title of each section using markdown headings (\# \#\#
\#\#\# \ldots). You can introduce each section using short sentences.
Then resources are presented in simple markdown tables as in
\href{basics_r.html}{the example page: useful resources to use R}.

Example of resources in a table:

\begin{longtable}[]{@{}
  >{\raggedright\arraybackslash}p{(\columnwidth - 2\tabcolsep) * \real{0.88}}
  >{\raggedright\arraybackslash}p{(\columnwidth - 2\tabcolsep) * \real{0.12}}@{}}
\toprule
Name & Links \\
\midrule
\endhead
\textbf{First steps in R} \emph{Institut Pasteur \& USR 3756} -- A very
basic tutorial to start R & \\
\textbf{Hands-On Programming with R} \emph{G Grolemund} -- A practical
guide to start programming with R & \\
\textbf{R studio cheatsheets} -- Very useful synthetic sheets to find
the function you need & \\
\textbf{Rstudio youtube channel} & \\
\bottomrule
\end{longtable}

\hypertarget{heading}{%
\subsection{Heading}\label{heading}}

The yaml (or heading) of each .Rmd file includes title and description
of the page that can be edited.

title: ``TITLE OF THE PAGE''

description: \textbar{}

DESCRIPTION OF THE PAGE

date: ``r Sys.Date()''

output: distill::distill\_article

\hypertarget{resources}{%
\subsection{Resources}\label{resources}}

In each section or sub-section, resources are included in markdown
tables. Each includes two columns: Name \& Links. the Name column
include the description of the resources. This description should
include the minimum information listed below. In the Links column,
resources are classified by type (course, book, website\ldots) and level
(begin, intermediate, expert, everyone) in order to guide readers
towards what they are looking for faster.

\hypertarget{type}{%
\paragraph{Type}\label{type}}

Types are classified according to the following table. Links to
resources can be included using the function
\emph{ilink(``type'',urladresslink,lvl)}. For each resource, please
follow this formating:

\begin{longtable}[]{@{}
  >{\raggedright\arraybackslash}p{(\columnwidth - 6\tabcolsep) * \real{0.16}}
  >{\raggedright\arraybackslash}p{(\columnwidth - 6\tabcolsep) * \real{0.44}}
  >{\raggedright\arraybackslash}p{(\columnwidth - 6\tabcolsep) * \real{0.32}}
  >{\raggedright\arraybackslash}p{(\columnwidth - 6\tabcolsep) * \real{0.09}}@{}}
\toprule
type & Name: resource formating & Links:code & Links \\
\midrule
\endhead
full course & \textbf{Title} \emph{Author} -- short description &
\emph{ilink(``course'', ``url'',lvl=0)} & \\
video & \textbf{Title} \emph{Author} -- short description &
\emph{ilink(``video'', ``url'',lvl=0)} & \\
open book & \textbf{Title} \emph{Author} -- short description &
\emph{ilink(``tuto'', ``url'',lvl=0)} & \\
book (not free) & \textbf{Title} \emph{Author} -- short description &
\emph{ilink(``book'', ``url'',lvl=0)} & \\
article & \textbf{Title} \emph{First Author et al.} -- year &
\emph{ilink(``article'', ``url'',lvl=0)} & \\
website & \textbf{Title} -- short description & \emph{ilink(``site'',
``url'',lvl=0)} & \\
practical & \textbf{Title} -- short description & \emph{ilink(``exo'',
``url'',lvl=0)} & \\
\bottomrule
\end{longtable}

\hypertarget{level}{%
\paragraph{level}\label{level}}

Levels are defined by color and can be attributed to any type of
resources, using the argument ``lvl'' of the function ilink().

\begin{longtable}[]{@{}
  >{\raggedright\arraybackslash}p{(\columnwidth - 6\tabcolsep) * \real{0.12}}
  >{\raggedright\arraybackslash}p{(\columnwidth - 6\tabcolsep) * \real{0.42}}
  >{\raggedright\arraybackslash}p{(\columnwidth - 6\tabcolsep) * \real{0.32}}
  >{\raggedright\arraybackslash}p{(\columnwidth - 6\tabcolsep) * \real{0.12}}@{}}
\toprule
level & i.e. & code for link & link \\
\midrule
\endhead
begin & resources to start with & \emph{ilink(``tuto'', ``url'', lvl=1)}
& \\
intermediate & resources to progress in the field &
\emph{ilink(``tuto'', ``url'', lvl=2)} & \\
advanced & resources for researchers already in the field &
\emph{ilink(``tuto'', ``url'', lvl=3)} & \\
everyone & resources for everyone & \emph{ilink(``tuto'', ``url'')} & \\
\bottomrule
\end{longtable}

\hypertarget{manage-and-modify-your-page}{%
\section{Manage and modify your
page}\label{manage-and-modify-your-page}}

\hypertarget{modify-your-page}{%
\subsubsection{Modify your page}\label{modify-your-page}}

1- Pull the repository from github.
(\href{https://happygitwithr.com/rstudio-git-github.html\#clone-the-test-github-repository-to-your-computer-via-rstudio}{see
here for help})

2- Create a branch to work on for your page, so you can work on it
before it appears online
(\href{https://aberdeenstudygroup.github.io/studyGroup/lessons/SG-T1-GitHubVersionControl/VersionControl/\#2.3.}{see
section 2.4.2 here for help}).

3- Install and source the package distill (and rmarkdown).

4- Open and edit the .Rmd file of your page. You never need to open the
.html file. (see rmarkdown
\href{https://www.rstudio.com/blog/the-r-markdown-cheat-sheet/}{cheatsheet}
\& \href{https://www.markdownguide.org/extended-syntax/\#tables}{tables}
for help)

4- Use the function rmarkdown::render\_site(``yourpage.Rmd'') to update
the associated .html file in the docs folder.

5- Save, commit and push on your branch
(\href{https://aberdeenstudygroup.github.io/studyGroup/lessons/SG-T1-GitHubVersionControl/VersionControl/\#2.3.}{see
section 2.4.3 here for help})

6- When your page is ready to be published online or has new
modifications that you want to appear online, ask for merging your
branch with the master branch using pull request.
(\href{https://aberdeenstudygroup.github.io/studyGroup/lessons/SG-T1-GitHubVersionControl/VersionControl/\#2.3.}{see
section 2.4.4 here for help})

7- Do not delete your branch as it will be used by others to suggest
modifications/new resources on your page.

\hypertarget{manage-your-page}{%
\subsubsection{Manage your page}\label{manage-your-page}}

Other contributors will probably ask you to add some resources on your
page. For this, they have two solutions: Use pull request or Fill the
online formulaire.

\hypertarget{using-pull-request}{%
\paragraph{Using pull request}\label{using-pull-request}}

If other contributors use pull requests, they should make pull requests
on your branch (names after the name of your page) and you will be
automatically notified by email. In the pull request, you will directly
see the modifications suggested on your page. Please check the
relevance, suitability and link of the resources suggested before
accepting the suggested changes. See here for documentation about how to
\href{https://docs.github.com/en/pull-requests/collaborating-with-pull-requests/reviewing-changes-in-pull-requests/about-pull-request-reviews}{review}
and
\href{https://docs.github.com/en/pull-requests/collaborating-with-pull-requests/incorporating-changes-from-a-pull-request/about-pull-request-merges}{accept}
a pull request.

\hypertarget{using-formulaire}{%
\paragraph{Using formulaire}\label{using-formulaire}}

This part is not ready yet!!!

If people use the formulaire online, you will receive an email including
the suggested resource to add on your page (only one by one). This email
should already include all needed information and appear in the right
format. So eventually, you will only have to check the resource and
modify your page by copy-paste the line to one of the tables in your
page.

<!--radix_placeholder_site_after_body-->
<!--/radix_placeholder_site_after_body-->

<!--radix_placeholder_navigation_after_body-->
<!--/radix_placeholder_navigation_after_body-->

\end{document}
